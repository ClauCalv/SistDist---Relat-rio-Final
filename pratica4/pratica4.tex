\subsection*{1. Barrier and Queue Tutorial.}
\addcontentsline{toc}{section}{1. Barrier and Queue Tutorial}

\subsubsection{1.1 - Modificar a variável ZK de acordo com a instalação do Zookeeper em run barrier.sh.}
\addcontentsline{toc}{subsection}{1.1 Modificar a variável ZK de acordo com a instalação do Zookeeper em run barrier.sh.}

\subsubsection{1.2 - O que significa a variável SIZE em run barrier.sh?}
\addcontentsline{toc}{subsection}{1.2 O que significa a variável?}

\subsubsection{1.3 - Executar run barrier.sh. O que aconteceu?}
\addcontentsline{toc}{subsection}{1.3 Executar run barrier.sh.}

\subsubsection{1.4 - Modificar a variável ZK de acordo com a instalação do Zookeeper em
run queue producer.sh.}
\addcontentsline{toc}{subsection}{1.4 Modificar a variável ZK de acordo com a instalação do Zookeeper em
run queue producer.sh.}

\subsubsection{1.5 - Modificar a variável ZK de acordo com a instalação do Zookeeper em run queue consumer.sh.}
\addcontentsline{toc}{subsection}{1.5 Modificar a variável ZK de acordo com a instalação do Zookeeper em run queue consumer.sh.}

\subsubsection{1.6 - O que significa a variável SIZE em run queue producer.sh? E em
run queue consumer.sh?}
\addcontentsline{toc}{subsection}{1.6 O que significa a variável SIZE em run queue producer.sh? E em run queue consumer.sh?}

\subsubsection{1.7 - Executar run queue consumer.sh e depois run queue producer.sh.
O que aconteceu? Alterne a execução dos scripts}
\addcontentsline{toc}{subsection}{1.7 Executar run queue consumer.sh e depois run queue producer.sh.}

\subsection*{2. Tutorial de lock baseado no Barrier and Queue Tutorial}
\addcontentsline{toc}{section}{2. ?}

\subsubsection{2.1 - Modificar a variável ZK de acordo com a instalação do Zookeeper em run lock.sh.}
\addcontentsline{toc}{subsection}{2.1 Modificar a variável ZK de acordo com a instalação do Zookeeper em run lock.sh.}

\subsubsection{2.2 - O que significa a variável WAIT em run lock.sh?}
\addcontentsline{toc}{subsection}{2.2 O que significa a variável WAIT em run lock.sh?}

\subsubsection{2.3 - Executar várias instâncias de run lock.sh. O que aconteceu?}
\addcontentsline{toc}{subsection}{2.3 Executar várias instâncias de run lock.sh}

\subsubsection{2.4 - Executar várias instâncias de run lock.sh e, em seguida, matar
alguma instância intermediária. O que aconteceu?}
\addcontentsline{toc}{subsection}{2.4 Executar várias instâncias de run lock.sh e, em seguida, matar alguma instância intermediária}